\documentclass[a4paper, 12pt]{article}
\usepackage{color}
\usepackage[UTF8]{ctex}
\usepackage{ulem}

\begin{document}

\title{CSP 2023 游记}
\author{flyingfrog}
\maketitle

\section{CSP-J}

\subsection{T1}

这边 10 分钟搞定

\subsection{T2}

比较简单的贪心,大样例一如既往地 CCF 风格 {\sout{(奇水无比)}},当场破掉大样例

{\sout{然后惨遭 CCF背刺挂60分}}

\subsection{T3}

乱搞一小时\quad 乱调大样例\quad 乱挂 20 分
{\Huge{模拟题死吧啊啊啊啊啊}}

\subsection{T4}

\begin{itemize}
    \item \textbf{关于 dijkstra:}
\end{itemize}

\sout{它死了} 对不起对不起是我错了(死不悔改)

我没有证贪心正确性,把 dijkstra 换成 spfa 好像就可以了,挂 15 分

\section{CSP-S}

\subsection{T1}

让我们来看看哪个 Joker 连普及题都没切出来?

啊!原来就是——\textbf{我}

100 → 40,1= → 2=

\subsection{T2}

似乎做过类似思路的题,考场切了

\sout{然后败在了 T1 手里}

\subsection{T3}

乱搞一坤时\quad 乱调大样例\quad 乱挂 100 分
{\Huge{模拟题死吧啊啊啊啊啊}}

\subsection{T4}

\sout{不是我 5 分钟乱写的代码都有 5 分}

\section{总结}

属于是把 1= 自己丢了,不过应该没什么大问题

\section{NOIP 计划}

\subsection{CSP 备赛问题}

本来是想赛前打打板子,但是没有笔记本打不了,别的没什么问题

\subsection{每周安排}

\begin{itemize}
    \item 周日:整天
    \item 周一:下午第三节课到吃饭
    \item 周二:
    \begin{itemize}
        \item 上午第四节课
        \item 下午第一节课
        \item 下午第四节课
    \end{itemize}
    \item 周三:上午第四节课到吃饭
    \item 周四:
    \begin{itemize}
        \item 上午第四节课到吃饭
        \item 下午第二节课
        \item 下午第四节课
    \end{itemize}
    \item 周五:
    \begin{itemize}
        \item 上午第四节课
        \item 下午第二节课
        \item 下午第四节课
    \end{itemize}
    \item 周六:上午第三节课之后
\end{itemize}

\subsection{希望老师做的}

\sout{能不能和whk老师商量下把作业减减或者推迟也行}

\Huge NOIP AK!!

\end{document}